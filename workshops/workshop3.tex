%%%%%%%%%%%%%%%%%%%%%%%%%%%%%%%%%%%%%%%%%
% Lachaise Assignment
% LaTeX Template
% Version 1.0 (26/6/2018)
%
% This template originates from:
% http://www.LaTeXTemplates.com
%
% Authors:
% Marion Lachaise & François Févotte
% Vel (vel@LaTeXTemplates.com)
%
% License:
% CC BY-NC-SA 3.0 (http://creativecommons.org/licenses/by-nc-sa/3.0/)
% 
%%%%%%%%%%%%%%%%%%%%%%%%%%%%%%%%%%%%%%%%%

%----------------------------------------------------------------------------------------
%	PACKAGES AND OTHER DOCUMENT CONFIGURATIONS
%----------------------------------------------------------------------------------------

\documentclass{article}

% Include the file specifying the document structure and custom commands

%----------------------------------------------------------------------------------------
%	ASSIGNMENT INFORMATION
%----------------------------------------------------------------------------------------

\title{MMOD101: Workshop ``Analyzing 3D PDB structures.''} % Title of the assignment

\author{Alexey K. Shaytan\\ \texttt{alex@intbio.org}} % Author name and email address

\date{Lomonosov Moscow State University --- \today} % University, school and/or department name(s) and a date

%----------------------------------------------------------------------------------------

\begin{document}

\begin{warn}[Requirements:]

1) Pymol \url{http://www.pymol.org} 

2) UCSF Chimera

3) VMD

4) Access to a Jupyter notebook evironment with Python 3, MDanalysis, nglview and prody libraries

5) 3-button mouse
\end{warn}

\tableofcontents

%use Pymol and UCSF Chimera programs for analysis
%\section{do analysis in Python using MDanalysis and Prody libraries}
%\section{select, extract and analyze atom coordinates and dihedral angles.}
%\section{add hydrogens to the structure}
%\section{identify hydrogen bonds}
%\section{identify protonation states}
%\section{assign charges to the atoms}
%\section{compute and analyze electrostatic surface}
%\section{build contact maps}
%\section{identify domains}
%\section{find homologous structures}
%\section{make structural alignment, compute RMSD}
%\section{generate sequence alignment from structural alignment}
%\section{identify residue conservation}
%\section{analyze dynamics using elastic networks models}




\section{Use Pymol and UCSF Chimera programs for analysis}

In my opinion Chimera is more comfortable and have all important tools for analisys.

You can select structure on right list in Chimera, or you can fetch it by ID in ``File'' section.

To select chains, atoms, etc you can do this in ``Select'' section

Also you can click ``Favorites'' > sequence and select for witch chain you want to open sequence








\section{do analysis in Python using MDanalysis and Prody libraries}

To download file ``3Dstruct\_analysis.ipynb'' you need to have access to Newton 

Or u can download it on our  \href{http://intbio.org/mol_model_course/workshops/3Dstruct_analysis/}{workshop} in \textbf{Jupiter notebook} section

It's very important that the Kernel(core that is chosen for the interpretation) is called moldyn. To change Kernel press \textbf{Kernel>change kernel} and select right one

In the beginig we have ``Hat'' where we can run code.

And have some helpfull commands
The first two lines, will reload libraries if they somehow change:

: \%load\_ext autoreload

: \%autoreload 2
 
And the next two lines will turn off warnings, that can disturb us:

: import warnings

: warnings.filterwarnings('ignore')


    If we want to download file, for comfort to download it right to the server, we can go to page that we are interested in, for example \href{https://www.rcsb.org/structure/1PKN}{}
Click display files and copy PDB Format link (https://files.rcsb.org/view/1PKN.pdb)

After that in Jupiter notebook we can run bash command(!wget (link)), that will download file to local directory \\

    Also we have useful library \textbf{``nglview''} for structure viewing directly in Jupiter notebook 

: import nglview as nv

: view = nv.show\_structure\_file('1PKN.pdb')

: view \\

    \textbf{``MDAnalysis''} library need for trajectory analysis.

: import MDAnalysis

: u=MDAnalysis.Universe('1PKN.pdb')

: s=u.select\_atoms('all')

: nv.show\_mdanalysis(s)\\


    \textbf{``Prody''} library need for dynamic and elastic network models analysis.

: import prody

: pkn = prody.parsePDB('1PKN.pdb')

: nv.show\_prody(pkn)\\







\section{Select, extract and analyze atom coordinates and dihedral angles.}

    \textbf{Let's assume that we want to select coordinates of CA atoms and extract posistions of C-alpha atoms in MDAnalysis}\\
    
: s=u.select\_atoms('protein and name CA')

: s.residues.atoms.names (you can press after ``s.'' TAB and chose one of the methods{} or commands in atom group) 
  These are basics:
\begin{itemize}
    \item \textbf{s.n\_atoms} - counts atom quantity 

    \item \textbf{s.center\_of\_mass\{\}} - center of mass coordinates (thos is metod and we need to write brakets \{\})

    \item\textbf{s.n\_residues} - amount of residues

\end{itemize}

: s.atoms.positions \\

%17:13 timecode

    \textbf{Let's make a histogram of Calpha distances} \\

: import numpy as np 

: c[1:]

: c=s.atoms.positions 

  dvec=c[1:]=c[0:-1] 
  
  d=np.array([np.sqrt(i[0]**2=+i[1]**2+i[2]**2)]) for i in dvec

: import seabon as sns          (seabon- library for graphics display )

: sns.distplot(d[d<5])

: s.radius\_of\_gyration \\


    \textbf{Extract and amalyze diheral (torsion) angels} \\

    MDAnalysis.analysis.dihedrals - sublibrary \\
    
    : from MDAnalysis.analysis.dihedrals import Dihedral
    
    : r=u.residues[1] 
    
    : r.phi\_selection().names       (function for any residue, that can select atoms that belongs to ``Ramachandran'' angles)
    
    : dih=Dihedral([r.phi\_selection()])
    
    : dih.run()
    
    : dih.angles \\
    







\section{Add hydrogens to the structure} \\

    Reduce command - let you add hydrogens to standart output stream 

: !reduce 1PKN.pdb \\

Also you can output data to file \\

: !reduce 1PKN.pdb > 1PKN\_H.pdb \\

1PKN\_H.pdb - file name \\

This programms better to install from anaconda, you need to make sure, that programms are installed to your correct OS channel \\









\section{Identify hydrogen bonds}\\


Go to \textbf{Chimera} click Tools - Surface/Binding Analysis - FindHBond \\







\section{Identify protonation states} \\

Use command ``!propka'' \\

: !propka 1PKN.pdb 

: !cat 1PKN.pka









%40:00
\section{assign charges to the atoms} \\

: pdb2pqr 1PKN.pdb 1PKN.pqr --ff amber \\

: !cat 1PKN.pqr \\







%41:00
\section{Compute and analyze electrostatic surface} \\

Go to \textbf{Chimera}

First of all you need to and charges click Tools - Structure Editing - PDB2PQR \\

We can choose force field : AMBER \\

So we have problem, that we have't \textbf{Executable location} \\

Let's make new conda enviroment \\

Go to commmand promt and write \\

> conda create --name apbs \\
>y

Than go to anaconda.org \\

Search for pdb2pqr \\

Choose schrodinger/ pdb2pqr, and install it (copy paste command) \\

Then search for apbs by schrodinger and install it (copy paste command) \\

After that choose download Path in \textbf{Executable location} and apply \\


So after these steps we can calculate and analyze electrostatic surface \\

Click Tools - Surface/Binding Analysis - APBS \\

If you want to color surface, first of all you need to create it \\

Select all then click Action - Surface - show \\

And in Surface color menu, we can color it by \textbf{electrostatic potential}







%49:50
\section{build contact maps} \\

If we have distance matrix, we can use from ``scipy'' package ``pdist'' function, that calculate distance matrix

: from scipy.spatial.distance import pdist,squareform \\

: dm=suareform(pdist(c)) \\

: dm.shape \\

: ax = sns.heatmap(np.where(dm<12,1,0))       

  ax.invert\_axis() \\
  (np.where - function able to edit elements in matrix)







%56:00
\section{Identify domains}

Here it is proposed to use various services for searching domains

1) In NCBI Protein database \herf{https://www.ncbi.nlm.nih.gov/}, select Protein and write pdb code (1pkn), then select \textbf{Chain A Pyruvate\_Kinase} and click on \textbf{Identify Conserved Domains}\\

2) Pfam database \href{http://pfam.xfam.org/}. Click on \textbf{view a structure} and write 1pkn, on opened page click on \textbf{Domain organisation}\\

3) SCOP \href{http://scop.mrc-lmb.cam.ac.uk/}\\
\\




%64:00
\section{Find homologous structures} \\

In NCBI data base, as an example we take 1PKN protein. \\

\href{https://www.ncbi.nlm.nih.gov/Structure/cdd/wrpsb.cgi?INPUT_TYPE=live&SEQUENCE=1PKN_A} \\

It is comfrotable, that CDD classified it as a single domain. \\

Click on + in \textbf{List of domain hits}, then click on \textbf{cd00288} \\

So wee see full set of sequences. \\

You can color them by Identity. \\

To analyze by conservation, you need to count score in some cases with the Chanon entropy and color the structure according to it. \\

So you can try to color it by consrvation in Chimera.

Open sequence, click on \textbf{Edit} and \textbf{add sequence}. Let's add  \href{https://www.ncbi.nlm.nih.gov/protein/1A3W_A?report=fasta}{1A3W\_A}. \\

Copy sequence and paste it in \textbf{Plain text} and press apply.

And then click on \textbf{Headers} and select \textbf{Conservation}. \\

After that click on \textbf{Structure} an \textbf{Render by conservation} and click Ok in opened window. \\




\section{Make structural alignment, compute RMSD}\\

Go to Chimera. So we see, taht we have similar structure 1A3W, download it to chimera. \\

We see 2 structures, that we want to align. Click on \textbf{Tools}, then \textbf{Structure comparison} an select \textbf{MatchMaker}. \\

Select wich structure you want alignt. In Chain pairing select \textbf{Best-align pair of chains} and press Ok \\





\section{Generate sequence alignment from structural alignment}

Now we have structural alignment. And we can reverse this procedure. Click on \textbf{Tools}, then \textbf{Structure comparison} an select \textbf{Match -> Align}. Select chains and press ok \\

Sequence alignment based on structures is higher quality. \\





\section{Analyze dynamics using elastic networks models}\\

We'll use prody. \\

Choose protein. \\

:prot=pkn.select('protein') \\

:anm, atoms = prody.calcANM(prot, selstr='calpha') \\

:slowest\_mode = anm[0] \\

:print( slowest\_mode ) \\

:print( slowest\_mode.getEigval(),round(3) ) \\

:prody.writeNMD('1pkn.nmd',anm[:3],atoms) \\

So then we open our file in VMD to visualise it.

Click \textbf{Extensions},then \textbf{Analysis} and select \textbf{Normal Mode Wizard}, and download file.






\end{document}